\begin{abstract} As embedded systems become more and more popular and accessible to the large public and
    the performance increases with every generation, new ideas and projects implying specific
    implementations are brought out requiring development and support. One such example is
    represented by the various soft processors already existing or in development in the public
    space. These processors are implemented using logic synthesis which translates a design
    described by a hardware description language (HDL) into a logic gate implementation. Both
    open-source and commercial soft processors have advantages and disadvantages in terms of
    cost, support, complexity, performance and the use-case for which each processor was
    designed. A detailed comparison will be made in this report for some of the most known
    solutions in order to create a general idea about the subject and to highlight the above-
    mentioned features. \end{abstract}

\begin{keywords} Soft processor, hardware description language (HDL), logic synthesis, FPGA, RISC, ARM, instruction set \end{keywords}
    