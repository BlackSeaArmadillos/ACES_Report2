\hspace{0.5cm} When it comes to the design of an embedded system, a series of approaches can be taken into
consideration depending on the initial requirements. Cost, computation speed, memory size,
the environment in which the system will function and other features play an important role
in the initial design. The final approach needs to be balanced in order to exploit the
advantages of the chosen solution and to make sure that the disadvantages present an
insignificant impact. One can implement the system around a hard-core microcontroller
which is a cheaper approach when it comes to developing complex algorithms in a short
period of time and when the execution speed and the energy efficiency presents important
aspects of the project. The control logic is much easier to implement, and a vast set of
programming skills are available due to the large online community, thus allowing small
companies and individuals to come up with innovative ideas. A second method that may be
used for implementation is a system on chip containing a hard processor which combines the
conventional programmable hard-core microprocessor and an FPGA to exploit the features of
both elements. An SoC with a soft processor which includes the same hardware as stated in
the previous point, but the FPGA is used to simulate the behavior of a microprocessor can be
used as a third method. Let’s further detail this approach in the next rows.
A soft-core processor is a reusable hardware module \cite{Kuznetsov2019} whose architecture and behavior can
be described using a synthesizable HDL code such as VHDL or Verilog, as opposed to the
hard-core processors which are baked into the silicon. The use of soft processors includes
considerable advantages:

\begin{itemize}
    \item Flexibility: the soft processor can be configured and customized (i.e. adding custom
instructions and/or components such as floating-point units, memory protection units,
error detection and correction encoders/decoders etc) in order to adapt the system
based on the user needs.
    \item Multiple cores: depending on the number of logic cells, the fpga can hold multiple
cores thus having the possibility of parallelizing some tasks.
    \item The soft cores are technology independent meaning that they can be loaded on every
FPGA board, thus they present a high tolerance to obsolescence.
    \item In the case of an SoC, distinct processes can be divided between the hard and the soft
CPU.
    \item Portability: the soft core can be ported to any FPGA, as long as it is not vendor
specific.
    \item The architecture and behavior are written at higher abstraction level using a hardware
description language; therefore it is much easier to understand and to create a design.
\end{itemize}

Some of the disadvantages of a soft-core processor are represented by the overall high
cost of an FPGA, lower computation speeds than an MCU due to the FPGA fabric, the higher
complexity of a hardware description language compared to a programming language such as
C and the less numerous online community.
These days a large number of soft processors can be found on the internet, whether it be
open-source platforms supported by individuals or commercial solutions provided by various
vendors. In the following chapters we will present in detail the available platforms,
mentioning important features regarding performance, number of stages, availability,
required FPGA size, use cases, community etc.., and a comparison between them will be
made at the end concluding with the most important aspects and the possible directions that
could be approached in the future works.