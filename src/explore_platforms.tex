\hspace{0.5cm} The multitude of soft processors available online allow the users to choose the best
solution suitable for their projects. In this chapter we will explore in depth these processors,
provide a comparison in the form of a table and we will highlight the important features in
terms of performance and availability.

\subsection{Open-source processors}

\hspace{0.5cm} In this subsection we will explore the open-source soft processors available to the
individuals or companies without a paid fee due to the accompanied license. The lack of an
accepted open-source hardware license determined the use of software licenses such as GNU
LGPL, GNU GPL or BSD. A large number of processors can be found on github or published 
on the OpenCores.org website. As stated in the wikipedia page \cite{11}, the Open Cores is a
community formed by individuals, universities or companies that develop open source
hardware components by using electronic design automation (EDA) tools similar to how
open source software is produced. The community comes up with a large variety and number
of processors each having distinct and important features that are worth mentioning in this
paper such as Amber, pAVR, Leon2/3/4 and BERI.

\hspace{0.5cm} \textbf{Amber} is an open source 32-bit RISC processor based on the ARM architecture based on the ARMv2a instruction set, made by Conor Santifort. The processor is compatible with the
ARMv2a instruction set, therefore it can be implemented without an ARM license being
supported by the GNU toolset. The project is available for download on OpenCores.org and
is part of the open-source hardware movement which aims to develop libraries that can be
used in projects for free \cite{3}. According to the website, it was created in 2010 and was
updated several times, the final one being in 2019, which resulted in a stable version. The
processor was developed in Verilog 2001 and implemented on the Xilinx Spartan 6 SP605
FPGA development board being optimized for FPGAs and provides a complete embedded
system including the core and several peripherals such as UART, timers and Ethernet MAC.
There is no reset logic, all registers being reset as part of FPGA initialization. Two versions
of the core were developed as part of the Amber project. The Amber 23 was designed with a
3-stage pipeline, a unified instruction and data cache, a 32-bit wishbone interface and is
capable of 0.75 DMIPS / MHz. Register based instructions execute in a single cycle, except
for instructions involving multiplication. Load and store instructions require three cycles. The
Amber 25 has a 5-stage pipeline, separate data and instruction caches, a 128-bit Wishbone
interface, and is capable of 1.05 DMIPS / MHz. Both cores implement the same ISA and are
software compatible. They do not contain a memory management unit so they can only run
the non-virtual memory variant of Linux and they have been verified by booting a 2.4 Linux
kernel. \cite{3}\cite{6}

\hspace{0.5cm} \textbf{pAVR} is an AVR architecture compatible 8-bit microcontroller designed in VHDL by
Doru Cuturela. It is a general representation of an AVR microcontroller which can be
configured to simulate most of the AVR family members. The core was originally designed
to obtain a powerful processor in terms of MIPS and concluded in a core 3x faster than the
original AVR core. The pAVR is an easily maintainable and highly customizable design due
to its modularity, code writing techniques and documentation. It has a 6-stage pipeline with 1
clock per instruction at peak (1.7 CPI typical) for most instructions, an estimated clock
frequency of 50 MHz with 28 MIPS or 50 MIPS at peak (meanwhile the Atmel’s cores run at
15-16 MHz with 10 MIPS). According to the Open Cores website the project was finished in
the summer of 2002 and was updated a few times with the current status being “Alpha” due
to the newly discovered bugs. \cite{7}

\hspace{0.5cm} \textbf{Leon2} is a 32-bit processor model compatible with the SPARC V8 architecture created by the European Space Agency (ESA) and written in VHDL. It is a configurable core due to
its modularity and has separate instruction and data caches, hardware multiplier and divider,
interrupt controller, debug support unit with trace buffer, 2x24-bit timers, 2xUART, power
down function, watchdog and 16-bit I/O ports. The model is synthesizable with most tools
and can be implemented on ASICs and FPGAs.
Leon2-FT was designed as a radiation tolerant processor to be included in the spacecraft’s
on-board computer and is the fault tolerant version of the Leon2 implementing features like triple modular redundancy for the flip-flops and protection for all the internal and external
memories, using EDAC and parity bits. During a mission in outer space, the cosmic radiation
may affect the data storage or data exchange between the various components by introducing
single event upset (SEU) errors. These events appear due to the ionized particles striking
nodes of logic elements and randomly changing bits, in a way that may prevent the proper
functionality of the system and thus concluding to the mission failure. The processor design
incorporates advanced error detection and correction (EDAC) techniques that enable reliable
delivery of data over unreliable communication channels in order to withstand arbitrary SEU
errors that may cause the unwanted loss of data. The processor was used in two satellite
missions: ESA’s Intermediate eXperimental Vehicle (IXV) and China’s Chang’e 4 lunar
lander. \cite{8}\cite{9}

\hspace{0.5cm} \textbf{Leon3}, just like Leon2, is a synthesizable VHDL, 32-bit, SPARC V8 compatible soft
processor and it is highly configurable. One major difference between the two processors is
the number of pipeline stages: Leon3 has 7 stages while Leon2 has 5. It is based on a Harvard
architecture by having separate instruction and data cache, containing features like hardware
multiply/divide, MAC units, memory management unit, AMBA 2.0 AHB bus interface,
provide debug capabilities, a maximum of 125 MHz on FPGA and 400 MHz on ASIC
technologies and a performance of 1.4 DMIPS/MHz. Leon3-FT is the fault tolerant version of
the Leon3 processor and includes similar methods for overcoming the single-event-upset
(SEU) errors as Leon2-FT. Additional features of the Leon3-FT compared to the Leon3 are
register file SEU error correction with 4 errors per 32-bit word, cache memory error
correction, autonomous error handling and no timing impact due to error correction and
detection. \cite{9}\cite{10}

\hspace{0.5cm} \textbf{BERI} (Bluespec Extensible RISC Implementation) is a soft-core processor implementing
64-bit RISC instruction set and designed by the University of Cambridge and SRI
International using Bluespec System Verilog which is a high-level functional programming
language for describing hardware designs. The processor implements a floating-point unit
(FPU), programmable interrupt controller (PIC) and multicore operation (still in progress)
and is able to boot the FreeBSD operating system which supports multitasking, TCP/IP
networking, security features and third party open-source applications. The processor was
tested on a Terasic DE4 board at 100 MHz, with CPI close to 1.0. The processor is used
mainly in research purposes such as processor multithreading, verification and other
instruction set level experimentations and the source code is available on the project website
also provided in \cite{4}.

\subsection{Commercial platforms}

\hspace{0.5cm} Commercial soft processors are available under a proprietary license that has to be
purchased in order to be used in a proprietary product. The vendors contributed to the soft
processor’s world with revolutionary designs.

\hspace{0.5cm} \textbf{Cortex M1} is the first ARM processor designed to be implemented on a FPGA and
includes support for synthesis tools. It is a 32-bit processor compatible with ARMv6-M
architecture based on the Thumb and Thumb2 instruction set with a 3-stage pipeline and
debug capabilities. As important features we can count the 0.88 DMIPS/MHz. The processor is supported by many FPGA boards such as Altera’s Cyclone and Stratix series, Xilinx
Spartan-3, Virtex-2,3,4 and Artix-7 and Microchip Fusion, ProASIC3L and other boards.

\hspace{0.5cm} \textbf{Nios II} is a 32-bit RISC processor designed specifically for the Altera FPGA family. It is
suitable for a wide range of embedded applications such as digital signal processing (DSP) or
system control. The processor is highly configurable by adding a predefined memory
management unit or by defining custom instructions or custom peripherals. The processor
family comes in different variants Nios II classic (fast, standard and economy) and Nios II
gen2 (fast and economy) each having specific features. Nios II/s (standard) version is a 5-
stage pipeline processor, with up to 256 custom instructions that can be defined by the user,
having a JTAG debug module which may include hardware breakpoints or real-time trace
with a performance of 250 MIPS.

\hspace{0.5cm}  \textbf{PowerPC 440} is a 32-bit RISC processor core belonging to the Power PC family and
based on the Power ISA instruction set. It was designed to be used in various applications
such as SoC microcontrollers, television decoders, network appliances, ASICs and FPGAs
applications, storage devices and supercomputers. It is a high-performance processor with a
7-stage pipeline, memory caches, with speeds up to 800 MHz. Xilinx Virtex-5 FXT FPGA
board can support 2 PowerPC 440 cores obtaining a frequency of 550 MHz. BRE440 Rad
Hard SOC includes an IBM licensed PowerPC 440 core and is designed to be used in
radiation-rich environments for deep space, nuclear, defense and medical applications. The
processor includes an FPU, memory controllers, cache, ethernet ports, serial ports, DMA
channels, EDAC controller and PCI interface supporting a speed of 133 MHz and 2
MIPS/MHz.