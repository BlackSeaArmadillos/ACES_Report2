\hspace{0.5cm} Embedded systems are complex systems containing hardware and software which are used to
process input data from the outside world into signals that usually control a machine. Our
daily life is filled with similar systems, from the simple ones controlling a washing machine
or a refrigerator to the more complex ones used as on-board computers for satellites, planes
and cars.
An embedded system contains one or more microcontroller units (MCUs), memories and
input/output peripheral devices (sensors) and are used to gather(input) data from the exterior
world, be it sound, light, magnetic field or acceleration forces, and output them as control
signals usually used to drive an actuator or to display information. The data processing is
performed by the microcontroller by executing the program stored in the memory. Since the
first commercial field programmable gate array (FPGA) became available in 1985 and with
the advance of modern technology, the performance has increased in such a manner that
nowadays it is possible and more convenient to simulate a fully functioning microprocessor
together with its peripherals on an FPGA board before the manufacturing process begins.

\subsection{Related works}
\hspace{0.5cm} The “Soft processors as a prospective platform of the future” presents important aspects
about the soft CPUs compared to the general-purpose CPUs and to the ASIC based
processors. The paper concludes by mentioning that the modern FPGAs are the most
promising platforms for supporting soft CPUs with a wide area of applications such as
general-purpose products with digital control, tablet and mobile computers, computer
terminals and personal computers with improved security and reliability. Vivek Jayakrishnan
and Chirag Parikh compared in their paper \cite{Jayakrishnan2019} the power consumption, speed and resource
utilization of two SoCs: a soft-core based design composed of a soft ARM Cortex M0
processor along with custom IP components and a hard-core based design containing a hard
ARM Cortex A9 processor connected to the IP modules (VGA, I/O, Bus) loaded on the
FPGA. Both designs run an oscilloscope application which inputs voltage signals via an ADC
and outputs the scaled values on a display. The study concluded by observing that the hard-
core design is much faster and utilizes less resources, but the power consumption is higher, in
contrast to observations made in other studies, maybe due to some features that allow the
Cortex M0 to set the sleep mode over the unused internal components. A detailed comparison
of several open-source and commercial soft processors based on their features was made by
Jason G. Tong, Ian D. L. Anderson and Mohammed A. S. Khalid \cite{tong2007}. They also provided some
use cases in the industrial and research applications for the selected platforms. Their paper
was the main source of inspiration and was the most relevant for the initial topic of this
report.
The next chapter was realized by combining all the information and knowledge gathered
from the above-mentioned articles and from topic related websites that are included in the
reference section.